\documentclass[12pt,a4paper, draft]{article}
\usepackage[utf8]{inputenc}
\usepackage[T1]{fontenc}


\usepackage{amsmath}
\usepackage{amssymb}

\usepackage{lmodern}
\usepackage{indentfirst}
\usepackage{hyperref}

\usepackage{enumitem}

\renewcommand{\labelenumii}{\theenumii}
\renewcommand{\theenumii}{\theenumi.\arabic{enumii}.}

\begin{document}

\section*{Testovacie scenáre}
\subsection*{Existencia súboru}
Otestovať existenciu súboru a jeho korektné otvorenie. Ak funkcia dosta\-ne cestu korektného súboru, so súborom sa dá ďalej pokračovať, v opačnom prípade funkcia súbor zahodí a nepokračuje sa v ďalšom spracovávaní.

\subsection*{Crawler}
Otestovať funkcionalitu Crawlera. Používateľ pridá nové súbory v používateľskom rozhraní. Crawler má nájsť novopridané súbory a aktualizovať databázu. 

\subsection*{Validnosť súboru}
Otestovať validitu súboru. Ak funkcia dostane validný súbor (je v požado\-vanom formáte), pokračuje sa ďalej v procese. V opačnom prípade ak funkcia dostane nevalidný vstup (súbor je v zlom formáte), ďalej sa nepokračuje a funkcia súbor zahodí.

\subsection*{Rozlišovanie linuxových a windowsových súborov}
Otestovať rozlišovanie medzi linuxovým a windowsovým súborom (rozdiel je v type súboru a v pár znakoch). Funkcia rozlíši, či dostala na vstup linu\-xový alebo windowsový súbor a následne sa súbor parsuje podľa linuxového alebo windowsového formátu.  

\subsection*{Databáza}
Otestovať pridávanie nových prvkov do databázy. Pridá sa nový korektný súbor a následne treba zistiť, či sa pridal aj do databázy. Pridá sa nový nekorektný súbor, následne treba skontrolovať, či sa údaje zo súboru nepridali do databázy.


\end{document}