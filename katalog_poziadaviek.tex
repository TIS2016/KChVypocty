\documentclass[12pt,a4paper, draft]{article}
\usepackage[utf8]{inputenc}
\usepackage[T1]{fontenc}


\usepackage{amsmath}
\usepackage{amssymb}

\usepackage{lmodern}
\usepackage{indentfirst}
\usepackage{hyperref}

\usepackage{enumitem}

\renewcommand{\labelenumii}{\theenumii}
\renewcommand{\theenumii}{\theenumi.\arabic{enumii}.}

\begin{document}


\begin{titlepage}
	\centering
	\par\vspace{1cm}
	{\scshape\LARGE Kvantovo-chemické výpočty \par}
	\vspace{1cm}

	{\huge\bfseries Katalóg požiadaviek  \par}
	\vspace{2cm}
	{\Large\itshape Jaroslav Ištok \par}
	{\Large\itshape Katarína Fabianová \par}
	{\Large\itshape Dušan Suja \par}
	{\Large\itshape Jerguš Adamec \par}
	\vfill
	{\large \today\par}
\end{titlepage}

\pagebreak
\tableofcontents
\pagebreak

\section{Úvod}
\subsection{Účel tohto dokumentu}
Tento dokument obsahuje požiadavky, ktoré bude aplikácia implementovať. Je určený pre každého, kto sa bude podielať na vývoji. Zadávateľ projektu na základe tohto dokumentu skontroluje, či výsledná aplikácia bude spĺňať všetky jeho požiadavky. Členovia vývojového tímu ho budú používať pri plánovaní a riadení procesu vývoja. Pomôže im pri ďalšom návrhu aplikácie a neskôr im ju umožní otestovať. Tento dokument nie je odborným textom, je zrozumiteľný rovnako pre vývojárov aj zadávateľa projektu. Všetky použité odborné termíny sú vysvetlené v sekcii Definície, akronymi a skratky.

\subsection{Rozsah projektu}
Projekt sa radí medzi stredne veľké projekty. Jeho vývoj bude prebiehať v časovom horizonte približne pol roka, pričom implementácia bude prebiehať v rozsahu jedného mesiaca. Aplikácia bude pozostávať z viacerých modulov. Napríklad crawler, lexer, parser, ORM a jednoduchého, no zároveň intuitívneho grafického používateľského rozhrania, v ktorom bude možné okrem zobrazenia informácií z databázy, aj zobraziť grafický náhľad molekuly na základe vyrátaného výpočtu.

\subsection{Definície, akronymi a skratky}
\begin{enumerate}
	\item Crawler - nástroj, ktorý prehľadáva adresáre na serveroch a hľadá v nich potrebné súbory
	\item Lexer - nástroj, ktorý analyzuje a kontroluje štruktúru textového sú\-boru, v našom prípade, výsledok analýzy molekuly
	\item Parser - nástroj, ktorý spracuje dáta z textového súboru
	\item ORM - nástroj na prístup k databáze z programovacieho jazyka
	\item mysql - relačná (sql) databáza na ukladanie údajov
	\item apache - webový server, na ktorom bude bežať grafické používateľské rozhranie aplikácie
\end{enumerate}

\subsection{Odkazy}
\begin{itemize}
	\item Príklad zobrazenia spracovaných dát vo webovej aplikácii: \ \url{http://neon.dpp.fmph.uniba.sk/qch_calcs/index.php}
\end{itemize}

\subsection{Prehľad zostávajucej časti dokumentu}
\begin{itemize}
	\item V kapitole 2 sa nachádza stručný popis funkcionality aplikácie.
	\item V podkapitole 2.1 je stručne popísané o aký typ aplikácie sa jedná
	\item V podkapitole 2.2 obsahuje informácie, z ktorých si čitateľ vie vytvoriť ucelenú predstavu o funkcionalite aplikácie
	\item V podkapitole 2.3 sú popísané typy používateľov aplikácie
	\item V podkapitole 2.4 sa nachádzajú právne obmedzenia projektu
	\item V podkapitole 2.5 sa nachádzajú konkrétne rozhrania systému s jeho okolím
	\item Kpitola 3 je najdôležitejšia kapitola a obsahuje zoznam požiadaviek na projekt aj s popisom
\end{itemize}


\section{Všeobecný popis projektu}
Aplikácia bude napĺňať databázu zozbieranými údajmi. Bude obsahovať výsledky kvantovo-chemických výpočtov a bude pravidelne aktualizovaná o nové dáta. Aktualizácia dát bude prebiehať automaticky alebo manuálne na vyžiadanie používateľa. Aplikácia bude prístupná len konkrétnym používateľom, ktorí sa budú prihlasovať pomocou prihlasovacieho mena a hesla. Prihlásení používatelia budú môcť vo webovom rozhraní vyhľadávať a filtrovať údaje na základe bázy, metódy výpočtu a ďalších parametrov. Webové rozhranie bude ponúkať možnosť vykreslenia jednoduchého náhľadu molekuly na základe výsledku výpočtu.

\subsection{Perspektívny pohľad na projekt}
Projekt bude mať otvorený zdrojový kód. Bude bežať na linuxovom serveri a bude poskytovať webové rozhranie.

\subsection{Funkcionalita výslednej aplikácie}
Výsledná aplikácia bude pracovať nasledujúcim spôsobom: 
Vyhľadá sú\-bory s výsledkami výpočtov, ktoré vygenerovali externé programy. Prehľadávať sa budú servery a konkrétne priečinky, ktoré budú uložené v konfiguračnom súbore. Obsah konfiguračného súboru bude mať pevnú štruktúru a bude sa dať editovať používateľmi. Nad súbormi prebehne jednoduchá syntaktická analýza, na základe ktorej sa vyradia súbory, ktoré neobsahujú potrebné informácie. Dáta z validných súborov sa rozparsujú na jednotlivé údaje, ktoré sa uložia do databázy.
V pravidelných časových intervaloch alebo na vyžiadanie používateľa sa databáza bude rozširovať o dáta z novopridaných alebo zmenených súborov. Používateľské rozhranie bude realizované ako webová stránka a umožní používateľovi vykonávať požadované operácie nad zozbieranými dátami v databáze, ako je napríklad pokročilé vyhľadávanie na základe rôznych kritérií, či možnosť jednoduchého vykreslenia molekuly na základe výsledkov výpočtu. Na prácu s aplikáciou postačí pripojenie na internet a webový prehliadač. Aplikácia bude zabezpečená heslom a na jej používanie sa bude potrebné prihlásiť.

\subsection{Charakteristika používateľov}
Aplikácia je určená pre chemikov, ktorí pracujú s veľkým množstvom dát z vypočítaných výpočtov a potrebujú v nich mať poriadok. Vzhľadom na povahu aplikácie, nebudú existovať špeciálne používateľské role.

\subsection{Všeobecné obmedzenia}
Aplikácia bude obsahovať údaje z kvantovo-chemických výpočtov mole\-kúl. Dáta nie sú tajné, ani chránené a nepotrebujú špeciálne zabezpečenie, preto bude aplikácia obsahovať iba jednoduché zabezpečenie pomocou používateľského mena a hesla.

\subsection{Predpoklady a závislosti}
Aplikácia bude bežať na linuxovom serveri. Bude využívať relačnú da\-tabázu na ukladanie údajov. Predpokladáme, že nebude mať veľké nároky na procesor a operačnú pamäť. Vzhľadom na povahu dát, ktoré budú v databáze uložené, bude jej veľkosť maximálne niekoľko desiatok megabajtov. Používateľské rozhranie bude bežať na webovom serveri.

\section{Zoznam špecifických požiadaviek na \- systém}

\begin{enumerate}
	\item { \bf Vyhľadávanie a spracovanie súborov s dátami } 
	\begin{enumerate}
		\item Aplikácia bude vyhľadávať súbory na serveroch, ktoré obsahujú výsledky výpočtov. Vyhľadávať bude na základe prípony v názve súboru. Konkrétne priečinky, ktoré sa budú prehľadávať budú špecifikované v konfiguračnom súbore, ktorý budú editovať používatelia.
		\item Nad zozbieranými súbormi prebehne jednoduchá syntaktická ana\-lýza. Tým sa vylúčia súbory, ktoré sú chybné, resp. neobsahujú požadované informácie.
		\item Dáta z validných súborov sa rozparsujú na tokeny, čiže konkrétne údaje, ktoré sa uložia do databázy vo forme tabuliek.
		\item Údaje v databáze budú aktualizované v pravidelných časových intervaloch alebo na vyžiadanie používateľa.
	\end{enumerate}
	\item { \bf  Používateľské rozhranie aplikácie } 
	\begin{enumerate}
		\item Po otvorení stránky sa zobrazí prihlasovacia obrazovka, kde je potrebné zadať prihlasovacie údaje.
		\item V prípade zabudnutia hesla, bude možnosť obnoviť heslo, po za\-daní emailu.
		\item Po prihlásení sa zobrazí rozhranie aplikácie, ktoré bude generované dynamicky na základe zvolených kritérií filtrovania údajov, bude obsahovať ovládacie prvky, konkrétne tlačidlo na odhlásenie sa z aplikácie, tlačidlo na aktualizácie údajov v databáze, jednoduchý vyhľadávací box, ktorý bude prehľadávať databázu na základe zvolených kritérií, možnosť editácie konfiguračných súborov a mož\-nosť vykreslenia náhľadu molekuly.
		\item Kritéria, podľa ktorých bude možné vyhľadávať, budú napríklad báza, metóda, stechometria (vzorec) molekuly, dátum a ďalšie.
		\item Údaje sa budú zobrazovať v prehľadnej tabuľke.
	\end{enumerate}
\end{enumerate}
\end{document}

