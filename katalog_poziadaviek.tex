\documentclass[12pt,a4paper]{article}
\usepackage[utf8]{inputenc}
\usepackage[T1]{fontenc}


\usepackage{amsmath}
\usepackage{amssymb}

\usepackage{lmodern}
\usepackage{indentfirst}
\usepackage{hyperref}

\usepackage{enumitem}

\begin{document}


\begin{titlepage}
	\centering
	\par\vspace{1cm}
	{\scshape\LARGE Kvantovo-chemické výpočty \par}
	\vspace{1cm}

	{\huge\bfseries Katalóg požiadaviek  \par}
	\vspace{2cm}
	{\Large\itshape Jaroslav Ištok \par}
	{\Large\itshape Katarína Fabianová \par}
	{\Large\itshape Dušan Suja \par}
	{\Large\itshape Jerguš Adamec \par}
	\vfill
	{\large \today\par}
\end{titlepage}

\pagebreak

\tableofcontents

\pagebreak

\section{Úvod}

\subsection{Účel tohto dokumentu}
Dokument obsahuje všetky požiadavky, ktoré bude aplikácia implementovať. Je určený pre všetkých, ktorí sa budú podielať na jej vývoji. Patrí sem zadávateľ, developeri, TODO dokoncit strana 26(23) v prezentacii

\subsection{Rozsah projektu}
Projekt sa radí medzi stredne veľké projekty a jeho vývoj bude prebiehať v časovom horizonte približne pol roka.
Aplikácia bude obsahovať crawler, lexer, parser a orm na prácu s databázou a jednoduché intuitívne GUI.
Bonusová funkcionalita: grafické zobrazenie jednotlivých analyzovaných molekúl.

\subsection{Definície, akronymi a skratky}
\begin{itemize}
	\item Crawler - nástroj, ktorý prelieza adresáre na serveroch a hľadá súbory
	\item Lexer - nástroj, ktorý analyzuje štruktúru súboru (dokumentu)
	\item Parser - Nástroj na spracovanie údajov zo súborov
	\item ORM - nástroj na prácu s databázou z programovacieho jazyka
\end{itemize}

\subsection{Odkazy}
\begin{itemize}
	\item item
\end{itemize}

\subsection{Prehľad zostávajucej časti dokumentu}
V nasledujúcich kapitolách nájdete rozširujúce informácie o projekte.
Všeobecný popis projektu, perspektívu projektu, podrobný popis funkcionality, účel projektu, charakteristiku cieľových používateľov projektu a iné.


\section{Všeobecný popis}


\subsection{Perspektíva projektu}
\begin{itemize}
	\item item
\end{itemize}

\subsection{Funkcionalita výslednej aplikácie}
Výsledná aplikácia bude pracovať nasledujúcim spôsobom: \
Vyhľadá súbory s dátatmi z meracích prístrojov na konkrétnych serveroch. Dáta zo súborov najskôr analyzuje, potom spracuje a uloží ich do databázy.
V pravidelných časových intervaloch, alebo na vyžiadanie používateľa, bude rozširovať databázu o dáta z novopridaných súborov. Aplikácia poskytne používateľovi jednoduché a intuitívne webové rozhranie, v ktorom bude možné vykonávať požadované operácie nad dátami z databázy, ako je napríklad pokročilé vyhľadávanie na základe rôznych kritérii či možnosť jednoduchécho vykreslenia molekuly na základe údajov z databázy. Webové rozhranie bude vedieť poskytnúť informácie o určitej molekule, či bola niekedy analyzovaná, akými metódami bola analyzovaná a podobne. Na prácu s aplikáciou postačí pripojenie na internet a webový prehliadač. Aplikácia bude chránená heslom a každý používateľ bude mať svoje prihlasovacie údaje, ktoré bude možné zmeniť. Nových používateľov bude mocť pridávať iba administrátor.

\subsection{Charakteristika používateľov}
\begin{itemize}
	\item item
\end{itemize}

\subsection{Všeobecné obmedzenia}
\begin{itemize}
	\item item
\end{itemize}

\section{Predpoklady a závislosti}


\section{Zoznam špecifických požiadaviek na systém}
\begin{enumerate}[label={[\arabic*]}]
	\item { \bf Vyhľadávanie súborov na serveroch } \\ \\
	Aplikácie bude vyhľadávať súbory na serveroch, ktoré obsahujú dáta o molekulách, z meracích zariadení.
	\item { \bf Analýza nájdených súborov } \\ \\
	Aplikácia bude analyzovať nájdené súbory, resp. bude zisťovať či majú požadovanú štruktúru na ďalšie spracovanie.
	\item { \bf Spracovanie údajov z nájdených súborov } \\ \\
	Aplikácia bude vedieť spracovať dáta z nájdených súborov 
	\item { \bf Uloženie údajov do databázy }\\ \\
	Aplikácia bude vedieť uložiť spracované dáta prehľadne do databázy
	\item { \bf Aktualizovanie atabázy o novopridané súbory } \\ \\
	Aplikácia bude v pravidelných časových intervaloch, alebo na vyžiadanie používateľa,  zisťovať prítomnosť novopridaných súborov, ktorých údaje ešte nie sú uložené v databáze a aktualizovať databázu o nové údaje.
	\item { \bf  Webové používateľské rozhranie }
	Aplikácia bude ovládaná pomocou jednoduchého webového používateľského rozhrania.
	\item { \bf Možnosť vykonávania operácii nad dátami v databáze }\\ \\
	Aplikácia bude poskytovať nástroje na prácu s údajmi uloženými v databáze. Bude medzi nimi vedieť vyhľadávať podľa rôznych kritérii, zisťovať informácie o tom, či bola konkrétna molekula analyzovaná, resp. či sa nachádza v databáze. Keď nájde požadovanú molekulu, vypíše, akými metódami bola naalyzovaná, koľko krát bola analyzovaná a podobne.
	\item { \bf Zabezpečenie aplikácie } \\ \\
	Na prístup do aplikácie bude nutné prihlásenie svojim používateľkým menom a heslom. Používateľov bude môcť pridávať iba administrátor. 
	
\end{enumerate}

\section{Dodatky}
\begin{itemize}
	\item item
\end{itemize}

\end{document}

